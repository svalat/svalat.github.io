%
% Deuxième année ISTY - Module Administration Système
%
% Textes de T.D.
%
% Historique
%   1999/07/07 : pda/dntt : création
%

\documentclass [a4paper,twocolumn,landscape] {report}

	\usepackage[utf8]{inputenc}
    \usepackage {verbatim}	% pour la macro "listing"
    \usepackage [9pt,a4,times] {my2e}
    \usepackage[margin=2cm,landscape,twocolumn]{geometry}

    \setlength {\parskip} {1mm}

    \pagestyle {myheadings}

    \newcommand {\utilitaire} [1] {\texttt {#1}}
    \newcommand {\fichier} [1] {\texttt {#1}}

    \newcommand {\marquerchapitre} [1]
    {
        \addcontentsline {toc} {chapter} {#1}
        \markboth {#1} {#1}
    }
    
    \newcommand {\chapitresanstitre} [1]
    {
        \cleardoublepage
	\marquerchapitre {#1}
    }

    \newcommand {\titrechapitre} [1]
    {
	\begin {center}
	    \Large \bf #1
	\end {center}
	\bigskip
	\bigskip
    }

    \newcommand {\chapitre} [1]
    {
        \chapitresanstitre {#1}
        \titrechapitre {#1}
    }

    \newcounter {td}
    \newcounter {question} [td]
    \renewcommand {\thequestion} {\arabic{td}.\arabic {question}}

    \newcommand {\td} [1]
    {
        \cleardoublepage
        \refstepcounter {td}
	\marquerchapitre {TP \thetd\ - 2013/2014 - #1}
	\titrechapitre {TP \thetd\ - #1}
    }

    \newcommand {\but}
    {
        \bigskip
        {\Large\bf But}
        \bigskip
    }

    \newcommand {\question}
    {
        \refstepcounter {question}
        \bigskip
%         \vspace {2mm}
        {\Large\bf Exercice \thequestion} % {\Large\bf Question \thequestion}
        \nopagebreak
        % \bigskip
        \vspace {2mm}
        \nopagebreak
    }

    \newcommand {\examen} [2]
    {
	\chapitresanstitre {Examen : #2}
	\input {#1}
    }

    \newcounter {tp}
    \newcounter {exercice} [tp]
    \renewcommand {\theexercice} {\arabic{tp}.\arabic {exercice}}

%     \renewcommand {\fig} [2]
%     {
%         \vspace {2mm}
%         \begin {center}
%         \input {#1}
%         \end {center}
%         \vspace {2mm}
%     }

    \newcommand {\listing} [1]
    {
	{\small\verbatiminput {#1}}
    }

% \renewcommand {\figps} [2] {
%     \begin {center}
%         \epsfig {figure=#1.ps, width=#2}
%     \end {center}
% }


\begin {document}

\refstepcounter {td}

\td {Configuration du système}

Lors du TP précédent, nous avons vu la procédure d'installation manuelle d'une distribution
Linux. Pour cette séance, nous allons poursuivre en finalisant la configuration de notre
système pour l'adapter à nos besoins et mettre en place l'environnement opérationnel
pour l'utilisateur.

\section* {Noyau et amorce}

\question

Gentoo est une distribution source vous devez donc recompiler votre propre noyau.
Si vous avez peu de RAM, téléchargez les sources du noyau sur le site officiel de Linux
(www.kernel.org) au format tar.gz. Ceux qui ont assez de RAM peuvent laisser 
\textit{emerge} installer les sources au format tar.xz comme décrit dans la documentation.

\question

Trouvez les commandes permettant de lister le matériel présent afin de savoir comment configurer
votre noyau, notamment les périphériques PCI, chipset et carte graphique.

\question

La configuration par défaut contient déjà tout le nécessaire pour une machine virtuelle. 
Vous devez simplement activer la compilation en statique des systèmes de fichiers que vous 
utilisez et le support de DEVTMPFS (listing 2.3 de la documentation). Afin d'accélérer la 
compilation et jouer avec les sources, désactivez le support du debuggage du noyau (kacking), 
le support de wifi et des Mac.

\question

Compilez puis installez le noyau et ses modes. Installez \textit{grub} puis générez son fichier de 
configuration (/boot/grub/grub.cfg) avec la commande introduite par grub2. Regardez le contenu
du fichier.

\question

Configurez le mot de passe root et installez \textit{syslog-ng} et \textit{logrotate} pour gérer 
les logs.

\question

Sortez du chroot, démontez les partitions et redémarrez sur votre installation.

\section* {Configuration}

Maintenant que les fichiers sont en place, configurez votre Linux pour qu'il démarre correctement
dans votre environnement.

\question

Configuez votre environnement : clavier, localisation utilisant fr\_FR.UTF-8, nom d'hôte,
heure locale, activation du client dhcp (on utilisera dhcpcd), montage des partitions.

\section* {Utilisateurs}

\question

Créez un utilisateur à votre nom, assurez-vous qu'il puisse effectuer des commandes d'administration
avec \textit{su}. Installez et configurez la commande \textit{sudo}.

\question

Activez les quotas pour votre utilisateur, limitez-le à 200 Mo et faites un test en tentant de créer
un fichier plus gros pour obtenir l'erreur.

\section* {Accès distant SSH}

Modifiez la configuration de votre machine virtuelle pour pouvoir
vous y connecter en ssh (redirection de port sur l'interface réseau). On utilisera le port local 
2222 sur l'hôte. Redémarrez la VM. Activez le service SSH au démarrage, démarrez-le manuellement et
testez la connexion.

\section* {Paquets manuels}

Il peut parfois être nécessaire d'installer des programmes sur une station alors que l'on n’est pas
administrateur. On procède alors généralement en recompilant manuellement le programme depuis ses
sources et en l'installant dans un dossier de son home.

\question

Téléchargez les sources de hwloc (http://www.open-mpi.org/projects/hwloc/) et installez-les
dans /home/\${USE}R/usr.

\question

Configuez les variables d'environnements pour pouvoir utiliser \textit{hwloc-ls} comme tout autre commande
sans devoir utiliser son chemin complet.

\section* {Problème de dimensionnement}

Sur un serveur il arrive fréquemment que les ressources allouées aux utilisateurs soient amenées à évoluer.
Ceci est notablement vrai pour l'espace disque qui tend à croître avec le temps. Avec le partitionnement
actuel, il est semble difficile de faire évoluer la taille des partitions sans procéder à une réinstallation
complète. Pour cela, les développeurs système ont mis au point le système LVM (Logical Volume Manager)
afin de pouvoir redimensionner plus facilement les partitions de notre système.

\question

Ne le faites pas tout de suite, mais supposez que la partition /home soit trop petite, comment procéderiez-vous ?
Idem pour / ?

\question

Faites une archive du contenu de /home et supprimez la partition pour la recréer sous une forme basée sur LVM, toujours
en ext4. Remettez en place les fichiers.

\question

Ajoutez un nouveau disque dur à votre VM, choisissez la taille que vous voulez. Faites en sorte d'étendre la partition
/home basée sur LVM sur ce deuxième disque dur pour augmenter sa taille.

\question

Quel est le danger du partitionnement tel que nous l'avons mis en place si l'on considère des disques durs physiques à 
la place de nos disques virtuels ?

\end {document}
