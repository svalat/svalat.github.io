%
% Deuxième année ISTY - Module Administration Système
%
% Textes de T.D.
%
% Historique
%   1999/07/07 : pda/dntt : création
%

\documentclass [a4paper,twocolumn,landscape] {report}

	\usepackage[utf8]{inputenc}
    \usepackage {verbatim}	% pour la macro "listing"
    \usepackage [9pt,a4,times] {my2e}
    \usepackage[margin=2cm,landscape,twocolumn]{geometry}

    \setlength {\parskip} {1mm}

    \pagestyle {myheadings}

    \newcommand {\utilitaire} [1] {\texttt {#1}}
    \newcommand {\fichier} [1] {\texttt {#1}}

    \newcommand {\marquerchapitre} [1]
    {
        \addcontentsline {toc} {chapter} {#1}
        \markboth {#1} {#1}
    }
    
    \newcommand {\chapitresanstitre} [1]
    {
        \cleardoublepage
	\marquerchapitre {#1}
    }

    \newcommand {\titrechapitre} [1]
    {
	\begin {center}
	    \Large \bf #1
	\end {center}
	\bigskip
	\bigskip
    }

    \newcommand {\chapitre} [1]
    {
        \chapitresanstitre {#1}
        \titrechapitre {#1}
    }

    \newcounter {td}
    \newcounter {question} [td]
    \renewcommand {\thequestion} {\arabic{td}.\arabic {question}}

    \newcommand {\td} [1]
    {
        \cleardoublepage
        \refstepcounter {td}
	\marquerchapitre {TD \thetd\ - 2013/2014 - #1}
	\titrechapitre {TD \thetd\ - #1}
    }

    \newcommand {\but}
    {
        \bigskip
        {\Large\bf But}
        \bigskip
    }

    \newcommand {\question}
    {
        \refstepcounter {question}
        \bigskip
%         \vspace {2mm}
        {\Large\bf Exercice \thequestion} % {\Large\bf Question \thequestion}
        \nopagebreak
        % \bigskip
        \vspace {2mm}
        \nopagebreak
    }

    \newcommand {\examen} [2]
    {
	\chapitresanstitre {Examen : #2}
	\input {#1}
    }

    \newcounter {tp}
    \newcounter {exercice} [tp]
    \renewcommand {\theexercice} {\arabic{tp}.\arabic {exercice}}

%     \renewcommand {\fig} [2]
%     {
%         \vspace {2mm}
%         \begin {center}
%         \input {#1}
%         \end {center}
%         \vspace {2mm}
%     }

    \newcommand {\listing} [1]
    {
	{\small\verbatiminput {#1}}
    }

% \renewcommand {\figps} [2] {
%     \begin {center}
%         \epsfig {figure=#1.ps, width=#2}
%     \end {center}
% }


\begin {document}


\td {Installation de base}

Ce TP sera dédié à la prise en main des différents aspects d’un système Linux. 
Pour ce faire, on se propose d’installer une machine virtuelle avec la distribution 
Gentoo (www.gentoo.org). Cette distribution a, tout comme Arch (archlinux.org), la 
particularité de ne fournir qu’un minimum d’automatisation pour laisser le plein 
contrôle à son utilisateur.

\section* {Initialisation de la machine virtuelle}
   
\question

Authentifiez-vous sur les postes windows avec les utilisateurs locaux \textit{user} en utilisant 
le mot de passe \textit{tp\_admin}. Lancez l’outil de virtualisation VirtualBox et 
créez une machine virtuelle nommée \textit{client-gentoo}.

Pour les ressources, allouez lui au minimum 160Mo de RAM, 12Go de disque et autant de CPU que
votre machine physique.

\question

Téléchargez le LiveCD minimal x86 sur les sites mirrors français de Gentoo 
(\textit{/releases/x86/current-iso/}). Montez ce CD dans votre machine virtuelle et 
démarrez là.

\section* {Préparation du disque}

Vous trouverez la documentation complète d'installation diponibles en français sur le site
de gentoo. Le présent sujet syntétise les étapes principales dont vous autrez besoin pour
ce TP mais ne vous privez pas de chercher les compléments et explications détaillées
de la documentation.

\question

Partitionnez votre disque avec \textit{fdisk} ou \textit{cfdisk}. Attention, Gentoo est une distribution
source, nous aurons donc besoin de place pour la partition /. Vous pouvez viser : 

% \textbf{/boot} : 100Mo en ext2
% , \textbf{swap} : 256Mo, \textbf{/} : 6Go en ext3/4, \textbf{/home} : 6Go en ext3/4.

\begin{itemize}
 \item \textbf{/boot} : 100Mo en ext2
 \item \textbf{swap} : 256Mo (2 fois la RAM jusqu'à 1 ou 2 Go)
 \item \textbf{/} : 6Go en ext3/4
 \item \textbf{/home} : 6Go en ext3/4
\end{itemize}

\question

Formattez les partitions en pensant à leur attribuer un label, regardez les options des commandes
de formattage pour cela.

\question

Montez votre partition principale sur \textit{/mnt/gentoo}. Créez y les dossiers \textit{boot} et \textit{home} pour 
ensuite y monter les deux autres partitions. Activez le swap.

\question

Allez dans ce dossier pour y télécharger l'archive \textit{stage 3} de Gentoo. Utilisez le 
navigateur \textit{links} pour cela. La touche \textit{d} permet de lancer le téléchargement 
d'un lien. Téléchargez également le dernier \textit{snapshot} de l'arbre des paquets 
(\textit{portage}).

\question

Extraire les archives à leur place respective (/mnt/gentoo et /mnt/gentoo/usr), n'oubliez
pas l'option \textit{-p} de tar pour l'archive \textit{stage 3}.

\section* {Configuration}

Maintenant que les fichiers sont en place, configurez votre Linux pour qu'il démarre correctement
dans votre environnement.

\question

Afin de travailler dans l'environnement Linux installé nous allons faire un \textit{chroot} vers
le dossier \textit{/mnt/gentoo}. Suivez la documentation (6a).

\question

Configuez votre environnement : clavier, localisation utilisant fr\_FR.UTF-8, nom d'hôte,
heure locale, activation du client dhcp (on utilisera dhcpcd), montage des partitions.

\question

Installez la commande \textit{htop} avec le gestionnaire de paquet (\textit{emerge}) pour 
surveiller l'utilisation des ressources pendant la compilation du noyau.

% \newpage
\section* {Noyau et amorce}

\question

Gentoo est une distribution source vous devez donc recompiler votre propre noyau.
Si vous avez peu de RAM, téléchargez les sources du noyau sur le site officiel de Linux
(www.kernel.org) au format tar.gz. Ceux qui ont assez de RAM peuvent laisser 
\textit{emerge} installer les sources au format tar.xz comme décrit dans la documentation.

\question

Trouvez les commandes permettants de lister le matériel présent afin de savoir comment configurer
votre noyau, notamment les périphériques PCI, chipset et carte graphique.

\question

La configuration par défaut contient déjà tout le nécessaire pour une machine virtuelle. 
Vous devez simplement activer la compilation en statique des systèmes de fichiers que vous 
utilisez et le support de DEVTMPFS (listing 2.3 de la documentation). Afin d'accélérer la 
compilation et jouer avec les sources, désactivez le support du debuggage du noyau (kacking), 
le support de wifi et des Mac.

\question

Compilez puis installez le noyau et ses modes. Installez \textit{grub} puis générez son fichier de 
configuration (/boot/grub/grub.cfg) avec la commande introduite par grub2. Regardez le contenu
du fichier.

\question

Configurez le mot de passe root et installez \textit{syslog-ng} et \textit{logrotate} pour gérer 
les logs.

\question

Sortez du chroot, démontez les partitions et redémarrez sur votre installation.

\section* {Utilisateurs}

\question

Créez un utilisateur à votre nom, assurez vous qu'il puisse effectuer des commandes d'administration
avec \textit{su}. Installez et configurez la commande \textit{sudo}.

\question

Activez les quotas pour votre utilisateurs, limitez le à 200 Mo et faites un test en tentant de créer
un fichier plus gros pour obtenir l'erreur.

\question

Modifiez la configuration de votre machine virtuelle pour pouvoir
vous-y connecter en ssh (redirection de port sur l'interface réseau). On utilisera le port local 
2222 sur l'hôte. Redémarrez la VM. Activez le service SSH au démarrage, démarrez le manuellement et
tester la connexion.

\section* {Paquets manuel}

Il peut parfois être nécessaire d'installer des programmes sur une station alors que l'on est pas
administrateur. On procède alors généralement en recompilant manuellement le programme depuis ses
sources et en l'installant dans un dossier de son home.

\question

Téléchargez les sources de hwloc (http://www.open-mpi.org/projects/hwloc/) et installez-les
dans /home/\${USZE}/usr.

\question

Configuez les variables d'environnements pour pouvoir utiliser \textit{hwloc-ls} comme tout autre commande
sans devoir utiliser son chemin complet.

\end {document}
