%
% Deuxième année ISTY - Module Administration Système
%
% Textes de T.D.
%
% Historique
%   1999/07/07 : pda/dntt : création
%

\documentclass [a4paper,twocolumn,landscape] {report}

	\usepackage[utf8]{inputenc}
    \usepackage {verbatim}	% pour la macro "listing"
    \usepackage [9pt,a4,times] {my2e}
    \usepackage[margin=2cm,landscape,twocolumn]{geometry}

    \setlength {\parskip} {1mm}

    \pagestyle {myheadings}

    \newcommand {\utilitaire} [1] {\texttt {#1}}
    \newcommand {\fichier} [1] {\texttt {#1}}

    \newcommand {\marquerchapitre} [1]
    {
        \addcontentsline {toc} {chapter} {#1}
        \markboth {#1} {#1}
    }
    
    \newcommand {\chapitresanstitre} [1]
    {
        \cleardoublepage
	\marquerchapitre {#1}
    }

    \newcommand {\titrechapitre} [1]
    {
	\begin {center}
	    \Large \bf #1
	\end {center}
	\bigskip
	\bigskip
    }

    \newcommand {\chapitre} [1]
    {
        \chapitresanstitre {#1}
        \titrechapitre {#1}
    }

    \newcounter {td}
    \newcounter {question} [td]
    \renewcommand {\thequestion} {\arabic{td}.\arabic {question}}

    \newcommand {\td} [2]
    {
        \cleardoublepage
        %\refstepcounter {td}
        \setcounter {td}{#1}
	\marquerchapitre {TP \thetd\ - 2013/2014 - #2}
	\titrechapitre {TP \thetd\ - #2}
    }

    \newcommand {\but}
    {
        \bigskip
        {\Large\bf But}
        \bigskip
    }

    \newcommand {\question}
    {
        \refstepcounter {question}
        \bigskip
%         \vspace {2mm}
        {\Large\bf Exercice \thequestion} % {\Large\bf Question \thequestion}
        \nopagebreak
        % \bigskip
        \vspace {2mm}
        \nopagebreak
    }

    \newcommand {\examen} [2]
    {
	\chapitresanstitre {Examen : #2}
	\input {#1}
    }

    \newcounter {tp}
    \newcounter {exercice} [tp]
    \renewcommand {\theexercice} {\arabic{tp}.\arabic {exercice}}

%     \renewcommand {\fig} [2]
%     {
%         \vspace {2mm}
%         \begin {center}
%         \input {#1}
%         \end {center}
%         \vspace {2mm}
%     }

    \newcommand {\listing} [1]
    {
	{\small\verbatiminput {#1}}
    }

% \renewcommand {\figps} [2] {
%     \begin {center}
%         \epsfig {figure=#1.ps, width=#2}
%     \end {center}
% }


\begin {document}

\td {3}{Routeur réseau (DHCP, DNS, NAT)}

Pour ce TP on se propose de mettre en place un serveur qui fera office de routeur 
sur notre réseau (\textit{istycorp.fr}). Nous utiliserons deux machines virtuelles. 
La première, nommée \textit{client1}, fera office de client et correspondra à la station 
de travail d'un employé. La machine \textit{router} sera notre point de travail principal
et devra fournir :
\begin{itemize}
  \item La configuration dynamique des IP au client et de leur résolution DNS.
  \item Un cache DNS.
  \item Le pond entre internet et le réseau interne.
\end{itemize}

\section* {Création des machines virtuelles}

Une machine virtuelle vous est fournie pour le TP avec une installation minimaliste Debian. 
Dupliquez-la pour créer votre client et votre routeur. Assurez-vous de configurer leurs cartes 
réseaux émulées (configuration dans VirtualBox) de sorte que :
\begin{enumerate}
 \item La machine \textit{router} doit avoir deux interfaces :
    \begin{enumerate}
     \item Une interface permettant l'accès à internet (NAT).
     \item Une interface sur le réseau interne simulé (réseau interne : istycorp).
    \end{enumerate}
  \item La machine \textit{client1} doit avoir une unique interface sur le réseau interne 
  et devra passer par le routeur pour accéder à internet.
\end{enumerate}

\question

Pourquoi ne connecte-t-on pas directement les machines clients sur le réseau internet ? 
Citez au moins trois raisons.

%[cout des IP, menace internet vers client, contrôle des flux sortant (eg. Torrent...)]

\question

Rappelez le rôle du serveur DHCP et des serveurs DNS.

\section* {Configuration manuelle}

Afin de bien comprendre le rôle des composants nous allons dans un premier temps configurer 
les IP manuellement, démarrez les deux machines virtuelles.

\question

Afin de faciliter le travail, on pourra installer un serveur SSH sur le routeur et ouvrir 
la redirection de port sur l'interface publique (dans VirtualBox) pour pouvoir s'y connecter.

\question

L'image fournie est configurée pour le client, commencez par changer le nom d'hôte de la machine 
et vérifiez le nom de domaine local.

\question

Configurez manuellement (\textit{ifconfig}) le client et le routeur de sorte à avoir :
\begin{enumerate}
 \item IP du routeur : 192.168.0.1.
 \item IP du client : 192.168.0.1.
\end{enumerate}

Testez le ping entre les deux clients et regardez les deux tables de routages avec la commande
\textit{route}. Expliquez.

\question

Configurez le DNS du client de manière similaire au routeur. 
Tester la résolution de nom et l'accès à internet ? Pourquoi ne marche-t-elle pas ?

\section*{Mise en place du NAT}

Afin de partager la connexion internet du routeur avec les clients, il nous faut mettre en place
le re routage des paquets et le support du NAT.

\question

Rappelez l'objectif du NAT. Que peut apporter ipv6 par rapport à ce problème ?

\question

Activez le routage des paquets sur le routeur en éditant le fichier /etc/sysctl.conf. 
Vous pouvez aussi activer ce dernier sans redémarrer en utilisant une des deux commandes :

\begin{verbatim}
sysctl -w net.ipv4.ip_forward=1
echo 1 > /proc/sys/net/ipv4/ip_forward
\end{verbatim}

\question

Activez le support du NAT sur le routeur. Ce protocole doit être activer dans \textit{iptables}
afin d'adapter les règles de routage des paquets.

\begin{verbatim}
iptables -t nat -A POSTROUTING -s 192.168.0.0/24 -o eth0 -j MASQUERADE
\end{verbatim}

On peut rendre cette règle persistante en installant le paquet \textit{iptables-persistent}
et en générant le fichier de configuration :

\begin{verbatim}
iptables-save > /etc/rules.ipv4
\end{verbatim}

\question

Sur le client, vous devez rediriger les paquets à destination d'internet vers le routeur, pour cela,
activez l'utilisation d'une passerelle :

\begin{verbatim}
route add default gw 192.168.0.1 eth0
\end{verbatim}

\question

Tester la connexion internet, vous pouvez visualiser le routage à l'aide de l'outil 
\textit{traceroute} sur le client ou bien avec \textit{tcpdump} sur le routeur.

\question

Si l'on considère des outils tels que \textit{tcpdump} ou \textit{wireshark} sur le routeur,
quelle remarque pouvez-vous faire quand à la sécurité des protocoles de communications 
non cryptées (ex. Http) ?

\question

Tester la connexion sur un serveur FTP (par exemple un site miroir de distribution Linux ou du noyau).
Pourquoi est-ce que le mode actif de FTP ne fonctionne pas ?

\section* {Configuration dynamique : DHCP}

En pratique on utilise habituellement le protocole DHCP pour configurer automatiquement les clients
de notre réseau.

\question

Installez isc-dhcp-server sur le routeur et configurez-le avec les paramètres suivants :
\begin{enumerate}
 \item Plage d'IP : 192.168.0.2 – 192.168.0.10
 \item Masque d'IP : 255.255.255.0
 \item Passerelle : 192.168.0.1
 \item Domaine : istycorp.fr
 \item DNS : Celui de l'université, regardez dans /etc/resolve.conf
\end{enumerate}

\question

Validez le fonctionnement en redémarrant le client.

\section* {Cache DNS}

\question

Installez le cache DNS \textit{dnsmasq} de sorte que ce dernier soit intercalé entre 
votre client et internet.

\question

Définissez proprement les résolutions de noms pour les machines du réseau internes 
(en éditant /etc/hosts sur le routeur). Vérifiez la propagation au client au travers 
du cache DNS.

\question

Ajoutez une entrée écrasant la résolution de nom de Google et renvoyant sur un autre site. 
Que pensez-vous de la sécurité des traductions d'adresse sur un réseau dont on n’est pas 
maitre ? Comment assurer la sécurité de la connexion dans ce cas ?

\section* {Bonus : serveur NFS}

\question

S'il vous reste du temps installez un serveur NFS et montez-le sur le client pour disposer d'un 
dossier home centralisé.

\question

Discutez la sécurité des droits utilisateurs dans cette configuration, les restrictions d'accès 
aux fichiers seront-elles nécessairement toujours vérifiées ?

\section* {Bonus : redirection SSH}

Installez un serveur ssh sur le client et tentez de le rejoindre. Pour cela il faut activer 
la redirection au niveau du NAT de la VM du routeur et du NAT du routeur lui même. Utilisez 
le port 2223 pour cela.

\end {document}
