%
% Deuxième année ISTY - Module Administration Système
%
% Textes de T.D.
%
% Historique
%   1999/07/07 : pda/dntt : création
%

\documentclass [a4paper,twocolumn,landscape] {report}

	\usepackage[utf8]{inputenc}
    \usepackage {verbatim}	% pour la macro "listing"
    \usepackage [9pt,a4,times] {my2e}
    \usepackage[margin=2cm,landscape,twocolumn]{geometry}

    \setlength {\parskip} {1mm}

    \pagestyle {myheadings}

    \newcommand {\utilitaire} [1] {\texttt {#1}}
    \newcommand {\fichier} [1] {\texttt {#1}}

    \newcommand {\marquerchapitre} [1]
    {
        \addcontentsline {toc} {chapter} {#1}
        \markboth {#1} {#1}
    }
    
    \newcommand {\chapitresanstitre} [1]
    {
        \cleardoublepage
	\marquerchapitre {#1}
    }

    \newcommand {\titrechapitre} [1]
    {
	\begin {center}
	    \Large \bf #1
	\end {center}
	\bigskip
	\bigskip
    }

    \newcommand {\chapitre} [1]
    {
        \chapitresanstitre {#1}
        \titrechapitre {#1}
    }

    \newcounter {td}
    \newcounter {question} [td]
    \renewcommand {\thequestion} {\arabic{td}.\arabic {question}}

    \newcommand {\td} [2]
    {
        \cleardoublepage
        %\refstepcounter {td}
        \setcounter {td}{#1}
	\marquerchapitre {TP \thetd\ - 2013/2014 - #2}
	\titrechapitre {TP \thetd\ - #2}
    }

    \newcommand {\but}
    {
        \bigskip
        {\Large\bf But}
        \bigskip
    }

    \newcommand {\question}
    {
        \refstepcounter {question}
        \bigskip
%         \vspace {2mm}
        {\Large\bf Exercice \thequestion} % {\Large\bf Question \thequestion}
        \nopagebreak
        % \bigskip
        \vspace {2mm}
        \nopagebreak
    }

    \newcommand {\examen} [2]
    {
	\chapitresanstitre {Examen : #2}
	\input {#1}
    }

    \newcounter {tp}
    \newcounter {exercice} [tp]
    \renewcommand {\theexercice} {\arabic{tp}.\arabic {exercice}}

%     \renewcommand {\fig} [2]
%     {
%         \vspace {2mm}
%         \begin {center}
%         \input {#1}
%         \end {center}
%         \vspace {2mm}
%     }

    \newcommand {\listing} [1]
    {
	{\small\verbatiminput {#1}}
    }

% \renewcommand {\figps} [2] {
%     \begin {center}
%         \epsfig {figure=#1.ps, width=#2}
%     \end {center}
% }


\begin {document}

\td {4}{Services SSH / Apache / MySQL / NFS / Tomcat}

Pour ce TP, on considère la mise en place d'un serveur hébergeant les services utiles à notre entreprise 
\textit{istycorp.fr}. Pour cela nous allons considérer la mise en place de :
\begin{itemize}
  \item ssh pour faciliter la gestion distante du serveur.
  \item Les services habituels Apache / PHP / MySql pour fournir l'hébergement d'un Wordpress.
  \item Un serveur NFS pour partager les dossiers home en réseaux.
  \item Un serveur tomcat pour héberger Jenkins.
\end{itemize}  

\section* {Intégration dans l'infrastructure}

Nous repartirons du réseau mis en place lors du TP 3 à savoir un routeur et un client. Nous ajouterons donc une machine
\textit{server.istycorp.fr}. De même, cette dernière sera connectée à internet au travers du routeur.

Dans la suite nous allons mettre en place divers services qui devront être exposés à l'extérieur par le routeur. Si rien n'est fait, les connexions
entrantes arriveront sur le routeur et ne trouveront aucune réponse à leur requête. Nous devons donc faire une redirection de port
sur ce dernier pour renvoyer les paquets entrant vers le serveur approprié en fonction du port cible.

Pour les services à mettre en place dans la suite vous devrez ajouter des règles au fichier \textit{/etc/iptables/rules.v4} du routeur dans la partie NAT:

\begin{verbatim}
-I PREROUTING -p tcp --dport {DEST_PORT} -i ${WAN} -j DNAT \
	--to {REDIRECT_TO_IP}:{REDIRECT_TO_PORT}
\end{verbatim}

Nous sommes dans des machines virtuelles, les ports doivent donc également être exposés à votre machine locale
en reconfigurant les cartes réseau de VirtualBox sur le routeur.

\section* {Service SSH}

\question

Installez le serveur SSH (\textit{openssh-server} sous Debian) sur le serveur et vérifiez son fonctionnement. Pensez à
activer le routage du port 22 (iptables du routeur et VirtualBox).

\question

Mettez en place vos clés SSH pour vous connecter avec ces dernières plutôt qu'avec un mot de passe pour 
le compte 'isty'. Pour ce faire, il faut copier votre clé publique dans le fichier \textit{.ssh/authorized\_keys}.
Sous Linux cette tâche peut être automatisée avec la commande :
\begin{verbatim}
ssh-copy-id {user}@{hostname}
\end{verbatim}

\question

Afin de sécuriser l'accès à notre serveur nous n'autoriserons que l'accès par clé SSH et interdirons les
connexions directes au compte root. Modifiez la configuration du serveur en conséquence (\textit{/etc/ssh/sshd\_config}).

\question

Mettez en place l'outil \textit{fail2ban} afin de bloquer les tentatives d'intrusions par forte brute sur le ssh.
Vérifiez que vous vous faites bien bannir en échouant plusieurs fois à vous connecter.

\question

Après vous êtes fait bannir, regardez la sortie de la commande \textit{iptables-save} et expliquez.

\question

Observez vos traces dans les fichiers \textit{/var/log/auth.log} et \textit{/var/log/fail2ban.log}.

\question

Retrouvez votre accès en relançant le service après avoir supprimé les lignes dans \textit{auth.log}.

\section*{Apache / PHP / Mysql / Wordpress}

Nous allons considérer ici que notre entreprise a besoin du service WordPress pour diffuser des informations
sur l'entreprise. Pour cela nous aurons besoin d'un serveur supportant PHP et d'une base de données MySQL.

\question

Installez Apache et PHP, vérifiez le fonctionnement. Comme pour SSH il faudra activer le transfert de port sur
le routeur et dans VirtualBox.

\question

Installez votre serveur MySQL. Choisissez le mot de passe de son compte root comme demandé par le gestionnaire de paquet.

\question

Sur les distributions de type debian on aime en générale utiliser le script \textit{mysql\_secure\_installation} 
pour désactiver le compte anonyme et limiter l'accès root à un usage local.

\question

Afin de nous faciliter la gestion de notre base de données, nous installerons l'application \textit{phpmyadmin}.

\question

Installez WordPress et vérifiez son fonctionnement.

\question

Validez le fonctionnement en activant SSL avec un certificat auto signé. Debian fournit déjà la configuration, vous
n'avez plus qu'à l'activer avec :
\begin{verbatim}
a2ensite default-ssl
a2enmod ssl
\end{verbatim}

\question

Quelle remarque pouvez-vous faire à propos de la sécurité des certificats auto signés ?

\section*{Serveur NFS}

Nous décidons de partager les dossiers des utilisateurs sur le réseau interne à l'entreprise afin qu'ils puissent y accéder
depuis n'importe quelle station de travail. 

\question 

Mettez en place et testez le sur une machine cliente.

\question

Discutez la sécurité des droits utilisateurs dans cette configuration, les restrictions d'accès 
aux fichiers seront-elles nécessairement toujours vérifiées ?

\section*{Jenkins}

Les développeurs de notre entreprise ont besoin d'une plateforme d'intégration continue pour leur travail, nous leur proposons
Jenkins. Ce service est implémenté en J2EE et nécessite donc la présence d'un serveur applicatif java tel que \textit{tomcat}.

\question

Installez tomcat et déployez y \textit{Jenkins}. Pour ce faire, installez l'archive \textit{war} sans l'extraire dans le dossier
\textit{/var/lib/tomcat/webapps}.

\question

Avant de lancer tomcat, créez le dossier /var/jenkins, changez ses droits pour qu'ils appartiennent à l'utilisateur et au groupe tomcat7. Ajoutez la ligne suivante au fichier 

\question

On préfère habituellement ne pas exposer directement les serveurs applicatifs, on les cache donc derrière des serveurs frontaux de type apache, réputés plus robustes.
Apache joue alors le rôle de \textit{reverse proxy}. Pour activer ce fonctionnement :
\begin{enumerate}
 \item Activer le connecteur AJP sur le port 8009 dans \textit{/var/lib/tomcat7/conf/server.xml}.
 \item Activer le module \textit{proxy} et \textit{proxy\_ajp} d'apache avec la commande \textit{a2enmod}.
 \item Créez un fichier de configuration \textit{proxy-jenkins.conf} dans \textit{/etc/apache/mods-available/} avec le contenu suivant :
\begin{verbatim}
<Proxy /jenkins>
	ProxyPass ajp://localhost:8009/jenkins
	ProxyPassReverse ajp://localhost:8009/jenkins
</Proxy>
\end{verbatim}
	\item Créez le fichier \textit{proxy-jenkins.load} :
\begin{verbatim}
# Depends: proxy_ajp
\end{verbatim}
	\item Activez ce module avec a2enmod
\end{enumerate}

Relancez le serveur et vérifiez l'accès au travers d'apache.

%http://www.jtips.info/index.php?title=Tomcat/Mod_proxy

\section*{Bonus : webmail roundcube}

Installez le webmail Roundcube pour accéder à votre mail étudiant.

\section*{Bonus : redmine}

Installez et configurez le service redmine.

\end {document}
