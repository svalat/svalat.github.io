%
% Deuxième année ISTY - Module Administration Système
%
% Textes de T.D.
%
% Historique
%   1999/07/07 : pda/dntt : création
%

\documentclass [a4paper,twocolumn,landscape] {report}

	\usepackage[utf8]{inputenc}
    \usepackage {verbatim}	% pour la macro "listing"
    \usepackage [9pt,a4,times] {my2e}
    \usepackage[margin=2cm,landscape,twocolumn]{geometry}
    \usepackage{hyperref}

    \setlength {\parskip} {1mm}

    \pagestyle {myheadings}

    \newcommand {\utilitaire} [1] {\texttt {#1}}
    \newcommand {\fichier} [1] {\texttt {#1}}

    \newcommand {\marquerchapitre} [1]
    {
        \addcontentsline {toc} {chapter} {#1}
        \markboth {#1} {#1}
    }
    
    \newcommand {\chapitresanstitre} [1]
    {
        \cleardoublepage
	\marquerchapitre {#1}
    }

    \newcommand {\titrechapitre} [1]
    {
	\begin {center}
	    \Large \bf #1
	\end {center}
	\bigskip
	\bigskip
    }

    \newcommand {\chapitre} [1]
    {
        \chapitresanstitre {#1}
        \titrechapitre {#1}
    }

    \newcounter {td}
    \newcounter {question} [td]
    \renewcommand {\thequestion} {\arabic{td}.\arabic {question}}

    \newcommand {\td} [2]
    {
        \cleardoublepage
        %\refstepcounter {td}
        \setcounter {td}{#1}
	\marquerchapitre {TP \thetd\ - 2013/2014 - #2}
	\titrechapitre {TP \thetd\ - #2}
    }

    \newcommand {\but}
    {
        \bigskip
        {\Large\bf But}
        \bigskip
    }

    \newcommand {\question}
    {
        \refstepcounter {question}
        \bigskip
%         \vspace {2mm}
        {\Large\bf Exercice \thequestion} % {\Large\bf Question \thequestion}
        \nopagebreak
        % \bigskip
        \vspace {2mm}
        \nopagebreak
    }

    \newcommand {\examen} [2]
    {
	\chapitresanstitre {Examen : #2}
	\input {#1}
    }

    \newcounter {tp}
    \newcounter {exercice} [tp]
    \renewcommand {\theexercice} {\arabic{tp}.\arabic {exercice}}

%     \renewcommand {\fig} [2]
%     {
%         \vspace {2mm}
%         \begin {center}
%         \input {#1}
%         \end {center}
%         \vspace {2mm}
%     }

    \newcommand {\listing} [1]
    {
	{\small\verbatiminput {#1}}
    }

% \renewcommand {\figps} [2] {
%     \begin {center}
%         \epsfig {figure=#1.ps, width=#2}
%     \end {center}
% }


\begin {document}

\td {5}{Authentification centralisée : LDAP}

Pour ce TP on se propose de mettre en place une authentification centralisée pour l'ensemble des services
de l'entreprise afin de faciliter la gestion des comptes pour l'arrivée et le départ des employées. Nous
donc intégrer LDAP dans :
\begin{itemize}
  \item Authentification Unix des postes.
  \item Service Jenkins.
  \item Service redmine.
  \item Résolution DNS.
\end{itemize} 

\section* {Rappels sur LDAP}

\question

Quel est l'objectif du protocole LDAP ? 

\question

Dans quel cadre est-il principalement utilisé ? 

\question

Est-il limité à cet usage ?

\question

À quoi doit-on faire attention en ce qui concerne les informations habituelles que l'on place dans LDAP ?

\question

Donner les particularités de LDAP comparé à une base de données classique de type MySQL.

\section* {Mise en place du serveur}

\question

Installez le serveur LDAP sur la machine virtuelle fournie (\textit{ldap-utils} et \textit{slapd}).

\question
 
Créez votre base de données pour votre domaine (\textit{dc=istycorp,dc=fr}) : 
 
\begin{verbatim}
dpkg-reconfigure -plow slapd  
\end{verbatim}

Regardez le contenu de la base actuelle avec \textit{slapcat}. Repérez les différents éléments importants.

\question

Créez le schéma de votre annuaire afin de structurer les deux branches habituelles \textit{people} et \textit{groups}.
Pour ce faire, créez un fichier du type (en remplaçant les champs nécessaires) :

\begin{verbatim}
dn: ou=people,{{TODO}}
objectClass: organizationalUnit
ou: people

dn: ou=groups,{{TODO}}
objectClass: organizationalUnit
ou: groups
\end{verbatim}

Vous pouvez alors charger ce contenu dans LDAP avec la commande :

\begin{verbatim}
ldapadd -x -D "cn=admin,{{TODO}}" -f create-struct.ldiff -W
\end{verbatim}

\question

Ajoutez un utilisateur dans la branche \textit{people} :

\begin{verbatim}
dn: cn={{LOGIN}},{{TODO}}
objectClass: top
objectClass: account
objectClass: posixAccount
objectClass: shadowAccount
uid: {{LOGIN}}
uidNumber: {{TODO}}
gidNumber: {{TODO}}
userPassword: {{TODO}}
gecos: {{FULL_NAME}}
loginShell: /bin/bash
homeDirectory: {{TODO}}
\end{verbatim}

\question

Ajoutez un groupe (\textit{user}) et ajoutez-y l'utilisateur précédent :

\begin{verbatim}
dn: cn={{NAME}},{{TODO}}
objectClass: top
objectClass: posixGroup
cn: {{NAME}}
memberUid: {{UID}}
gidNumber: 1000
\end{verbatim}

\question

Vérifiez le bon fonctionnement de l'authentification avec cet utilisateur en listant ses informations 
avec une requête de type \textit{ldapsearch} :

\begin{verbatim}
ldapsearch -x -D "cn={{USER}},{{TODO}}" -W -b"cn={{USER}},{{TODO}}"
\end{verbatim}

\question

Changez le mot de passe de votre utilisateur avec la commande \textit{ldappasswd} :

\begin{verbatim}
ldappasswd -x -D 'cn={{USER}},{{TODO}}' -W -S
\end{verbatim}

\section* {Gestion graphique : phpldapadmin}

\question

Afin de se faciliter la vie, il est possible d'utiliser l'interface graphique \textit{phpldapadmin}.
Installez le paquet debian et connectez vous-dessus (vérifier la redirection du port de votre VM).

\question

Pensez à adapter le fichier \textit{/etc/phpldapadmin/config.php} pour que l'éditeur travail sur
le bon chemin LDAP (dc=....).

\question

Créez un deuxième utilisateur avec l'interface et ajoutez-le au groupe.

\section* {Intégration à PAM}

Nous allons maintenant intégrer LDAP dans l'authentification Unix de nos postes de travail. Afin de ne pas compliquer
la tâche, nous allons effectuer l'opération directeur sur la VM fournie.

\question

Installez les paquets nécessaires au support LDAP pour PAM et NSS: \textit{libpam-ldapd} et \textit{libnss-ldapd}..

\question

Apt-get a fait le travail pour nous, mais cherchez les lignes de configuration LDAP dans les fichiers de configuration
du dossier /etc/pam.d et dans /etc/nslcd.conf.

\question

Configurez PAM pour qu'il crée automatiquement les dossiers utilisateurs lors de leur première authentification.
Éditez le fichier \textit{/etc/pam.d/common-session} :

\begin{verbatim}
session     required      pam_mkhomedir.so skel=/etc/skel umask=0022
\end{verbatim}

\question

On pourra vérifier le bon fonctionnement entre NSS et LDAP avec la commande :

\begin{verbatim}
getent passwd
\end{verbatim}

\textit{En cas d'erreur, vous pouvez arrêter le service \textit{nslcd} et lancer le démon (\textit{nslcd}) à 
la main en mode debug (option -d) pour obtenir rapidement les messages d'erreurs.}

\question

Vérifiez le bon fonctionnement de l'authentification vers votre utilisateur par ssh ou en local dans la VM.

\pagebreak 

\section* {Intégration à divers services}

\question

Intégrez LDAP dans le service \textit{Jenkins} installé sur le serveur. On fera en sorte que seuls
les utilisateurs appartenant au groupe \textit{Jenkins} soient pris en compte.

\question

Gérez les noms d'hôtes locaux de votre réseau avec LDAP. On pourra s'inspirer de la documentation d'Archlinux :
\url{https://wiki.archlinux.org/index.php/LDAP\_Hosts}.

\question

Intégrez LDAP dans le service \textit{redmine} installé sur le serveur.

\section*{Filtrage par groupe dans PAM}

Activez le filtrage pour n'autoriser l'authentification Unix que pour les utilisateurs membres du groupe \textit{Unix}.
On pourra s'aider de l'aide Debian : \url{https://wiki.debian.org/LDAP/PAM}.

\end {document}
