%
% Deuxième année ISTY - Module Administration Système
%
% Textes de T.D.
%
% Historique
%   1999/07/07 : pda/dntt : création
%

\documentclass [a4paper,twocolumn,landscape] {report}

	\usepackage[utf8]{inputenc}
    \usepackage {verbatim}	% pour la macro "listing"
    \usepackage [9pt,a4,times] {my2e}
    \usepackage[margin=2cm,landscape,twocolumn]{geometry}
    \usepackage{hyperref}

    \setlength {\parskip} {1mm}

    \pagestyle {myheadings}

    \newcommand {\utilitaire} [1] {\texttt {#1}}
    \newcommand {\fichier} [1] {\texttt {#1}}

    \newcommand {\marquerchapitre} [1]
    {
        \addcontentsline {toc} {chapter} {#1}
        \markboth {#1} {#1}
    }
    
    \newcommand {\chapitresanstitre} [1]
    {
        \cleardoublepage
	\marquerchapitre {#1}
    }

    \newcommand {\titrechapitre} [1]
    {
	\begin {center}
	    \Large \bf #1
	\end {center}
	\bigskip
	\bigskip
    }

    \newcommand {\chapitre} [1]
    {
        \chapitresanstitre {#1}
        \titrechapitre {#1}
    }

    \newcounter {td}
    \newcounter {question} [td]
    \renewcommand {\thequestion} {\arabic{td}.\arabic {question}}

    \newcommand {\td} [2]
    {
        \cleardoublepage
        %\refstepcounter {td}
        \setcounter {td}{#1}
	\marquerchapitre {TP \thetd\ - 2013/2014 - #2}
	\titrechapitre {TP \thetd\ - #2}
    }

    \newcommand {\but}
    {
        \bigskip
        {\Large\bf But}
        \bigskip
    }

    \newcommand {\question}
    {
        \refstepcounter {question}
        \bigskip
%         \vspace {2mm}
        {\Large\bf Exercice \thequestion} % {\Large\bf Question \thequestion}
        \nopagebreak
        % \bigskip
        \vspace {2mm}
        \nopagebreak
    }

    \newcommand {\examen} [2]
    {
	\chapitresanstitre {Examen : #2}
	\input {#1}
    }

    \newcounter {tp}
    \newcounter {exercice} [tp]
    \renewcommand {\theexercice} {\arabic{tp}.\arabic {exercice}}

%     \renewcommand {\fig} [2]
%     {
%         \vspace {2mm}
%         \begin {center}
%         \input {#1}
%         \end {center}
%         \vspace {2mm}
%     }

    \newcommand {\listing} [1]
    {
	{\small\verbatiminput {#1}}
    }

% \renewcommand {\figps} [2] {
%     \begin {center}
%         \epsfig {figure=#1.ps, width=#2}
%     \end {center}
% }


\begin {document}

\td {6}{Sauvegarde et restauration}

Pour ce TP on se propose de mettre en place le système de backup et de restauration de notre VM serveur.
En l'état, le serveur d'istycorp héberge :
\begin{itemize}
  \item Une base de données mysql.
  \item Les homes des utilisateurs.
  \item Une base LDAP.
  \item Un service web Wordpress.
\end{itemize} 


L'objectif est de mettre en place une solution de backup automatisée et régulière de l'ensemble des données de ces services.
Vous devrez dans un second temps, restaurer ces données dans une machine virtuelle similaire disposant des services configurés
sans leurs données.

À l’issue de ce TP, il sera demandé de rendre la machine virtuelle restaurée ainsi qu'une copie de vos scripts crontab et script de backup. 
Un rapport sera également attendu dans lequel vous devrez rappeler et expliquer la procédure de backup retenue (commandes et explications) 
ainsi que les réponses aux questions "théoriques" posées par le sujet. Ce document sera à rendre avant jeudi 3 avril.


\section* {Stratégie de sauvegarde}

La mise en place de backup suggère un choix de stratégie visant à organiser les traitements à effectuer et décider
de leur régularité. Ce choix doit se faire, en fonction : des volumes, du taux de changement des données manipulées 
, du coût des supports de stockage et de l'impact éventuel sur l'activité des utilisateurs.

\question

Expliquez la notion de sauvegarde incrémentale.
\textit{(rapport)}

\question

Décidez d'un planning de sauvegarde pour notre serveur en incluant au moins une sauvegarde incrémentale. 
Justifiez vos choix horaires.

\question

Identifiez les contenus à sauvegarder, estimez grossièrement leurs volumes. Donnez sans entrer dans les détails des options,
les méthodes de sauvegarde de ces dernières.

\question

Rappelez et discutez rapidement les avantages et inconvénients des différents supports de sauvegarde disponibles.
\textit{(rapport)}

\question

À quoi doit-on faire attention sur la manière de stocker les supports de sauvegarde ? Citez au moins 4 points.
\textit{(rapport)}

\section* {Espace de stockage}

Afin de nous faciliter la vie, nous allons effectuer les backups sur un second disque dur que vous ajouterez à votre machine
virtuelle.

\question

Ajoutez un disque dur de taille suffisante à votre machine virtuelle.

\question

Formattez-le et montez le système de fichier au démarrage de la machine.

\question

Critiquez cette méthode de stockage.
\textit{(rapport)}

\section* {Sauvegarde}

Mettez en place un script effectuant les backups, ce dernier devra être appelé selon votre planning par cron. 

\question

Décidez d'une organisation des fichiers de sauvegarde en fonction de votre planning et des rotations nécessaires.
Il est conseillé ici de créer un dossier par (à vous de choisir sa nomenclature) sauvegarde et d'y placer les 
fichiers/dossiers des différents contenus à sauver.

\question

Écrivez le code de backup des fichiers utilisateurs. Utilisez la méthode de votre choix (tar ou dump). Attention
à préserver les propriétés (droits, dates...) des fichiers des utilisateurs.

\question

Sans le mettre en pratique ici, à quoi faudrait-il faire attention si l'on désire faire une seconde copie de ces 
fichiers de backup sur un CD/DVD ?
\textit{(rapport)}

\question

Écrivez le code de backup des bases de données MySQL.

\question

Dicutez le problème de cohérence des backups de base de donnée, quelles solutions existent ?
\textit{(rapport)}

\question

Quel problème peut-on rencontrer si notre base de données MySQL occupe plusieurs Go ? Sans les mettre en pratique, proposez des solutions.
\textit{(rapport)}

\question

Écrivez le code de backup de la base LDAP.

\question

À la fin des sauvegardes, générez un checksum (MD5 ou SHA) des fichiers de sauvegarde pour permettre de vérifier leurs intégrités
en cas de nécessité.

\question

Configurez les règles cron pour appeler votre script selon votre planning. On pourra valider le fonctionnement de la chaine
en mettant dans un premier temps des règles appelées toutes les 2-3 minutes.

% \pagebreak 
\section* {Restauration}

\question

Générez un premier backup selon votre stratégie :
\begin{itemize}
 \item Générez un backup entier 
 \item Exécutez le script next\_day.sh disponible dans le dossier /root
 \item Générez un backup incrémental avec votre script
\end{itemize}

\question

Tranférez les dossiers de backups nécessaires dans la seconde VM et effectuez la restauration.

\question

L'utilisateur "raj" vous contacte par mail car il a supprimé par mégarde le dossier "htop-dev" sur lequel il travaillait le premier jour.
Restaurez son dossier à partir des backups sans écraser les fichiers des autres utilisateurs.

\section* {Bonus}

\question

Utilisez LVM pour assurer la cohérence de la sauvegarde des fichiers homes grâce à son support des snapshots.


\end {document}
